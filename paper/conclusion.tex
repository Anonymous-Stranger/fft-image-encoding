After implementing the Cooley-Tukey and Naive DFT algorithms, we have come to the conclusion that Cooley-Tukey is far faster, and marginally more accurate than the Naive DFT. We have also found that our compression algorithm, while it achieves low root mean squared error, it compresses images extremely poorly. As shown in the pictures, despite achieving low RMSE, the pictures have many artifacts and don't always resemble the original image very well. Also, thanks to the fact that the stored results are floating point instead of integer, and have a complex and real part, unless high compression ratios are used, the resuling file is larger than the input. This means that unless extremely high compression ratios are used, making the image look horrible, it is impossible to shrink it effectively with our algorithm. There may be possible improvements to the algorithm, like replacing the higher frequencies with different values (instead of zero), but we didn't have the time to look into anything like this.
