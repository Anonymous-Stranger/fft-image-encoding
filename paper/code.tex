\lstset{language=Python,basicstyle=\footnotesize\ttfamily}

Below are selected portions of our implementation. Full code can be found at: \\ \href{https://github.com/Anonymous-Stranger/fft-image-encoding}{https://github.com/Anonymous-Stranger/fft-image-encoding}

\subsection{Naive DFT}
	\label{code:naive_dft_python}
	From \verb|dft.py|:
	\begin{lstlisting}
import numpy as np
def dft(pts, k_max=None):
    k_max = k_max or len(pts)
    return np.array([dftk(pts, k) for k in range(k_max)])
def dftk(pts, k):
    coeff = -2*np.pi*k/len(pts)
    return sum((x * np.exp(complex(0, coeff*n)) for n, x in enumerate(pts)))
	\end{lstlisting}

\subsection{2D DFT}
	\label{code:2d-dft}
	From \verb|fft2d.py|:
	\begin{lstlisting}
def fft2d(fft_algo, data):
    return np.transpose(np.array([fft_algo(x) for x in np.transpose(
        np.array([fft_algo(x) for x in data]))]))


def ifft2d(fft_algo, data):
    return np.transpose(np.array([ifft(fft_algo, x) for x in np.transpose(
        np.array([ifft(fft_algo, x) for x in data]))]))


def ifft(fft_algo, data):
    return fft_algo(np.concatenate(
        ([data[0]], np.flip(data[1:], 0)))) / data.size
	\end{lstlisting}

\subsection{Cooley-Tukey}
	\label{code:cooley_tukey_python}
	From \verb|cooley_tukey.py|:
	\begin{lstlisting}
import numpy as np
def cooley_tukey(data):
    if(data.size==1): return data
    even = cooley_tukey(data[::2])
    odd = cooley_tukey(data[1::2])
    multiplier = np.exp(np.pi*-2j/data.size)
    k = 1
    for i in range(0,odd.size):
        odd[i] *= k
        k *= multiplier
    return np.concatenate((even+odd,even-odd))
	\end{lstlisting}


\subsection{Compression}
	\label{code:compression}
	From \verb|fft2d.py|:
	\begin{lstlisting}
def encode_and_compress(fft_algo, max_term):
    def encoder(img):
        enc = fft2d(fft_algo, img)
        bound, mx = enc.shape[0], min(enc.size, max_term)
        return np.array([enc[x, y] for x, y in zigzag(bound, mx)])
    return encoder

# continued

def uncompress_and_decode(fft_algo, block_size):
    def decoder(data):
        full = np.zeros((block_size, block_size), dtype=np.complex64)
        for (x, y), v in zip(zigzag(block_size, len(data)), data):
            full[x, y] = v
        return ifft2d(fft_algo, full)
    return decoder

def zigzag(bound, max_term):
    """ Loops through a bound x bound array in a zigzag order up to max_term.
        For bound=4, max_term=14, it yields the positions of the numbers:
                0   2   3   9
                1   4   8   10
                5   7   11
                6   12  13
    """
    bound -= 1
    row, col, dir = 0, 0, -1
    for _ in range(max_term):
        yield row, col
        if (col == 0 and dir < 0) or (col == bound and dir > 0):
            row += 1
            dir = -dir
        elif row == 0 and dir > 0 or (row == bound and dir < 0):
            col += 1
            dir = -dir
        else:
            row -= dir
            col += dir
	\end{lstlisting}
