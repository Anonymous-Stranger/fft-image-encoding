The Fast Fourier Transform (FFT) is one of the most important numerical algorithms of our time. First discovered by Gauss in 1805, and later rediscovered by Cooley and Tukey in 1965, the fast fourier transform allows us to compute the discrete fourier transform of a set of points in $O(Nlog(N))$ time instead of $O(N^2)$ time, as the naive algorithm would do. Various different fast fourier transform algorithms have since been discovered, and the fast fourier transform has been used in all kinds of applications, from signal processing to data compression.

In this paper, we review a brief history of the FFT, show its mathematical foundations, and then present our implementation of the transform for image compression. We use two algorithms, a naive Discrete Fourier Transform and the Cooley-Tukey FFT, and for these algorithms show that they are equally effective and correct, and that Cooley-Tukey is more performant than the naive DFT. Finally, we consider some of the limitations of our implementation, specifically the limited compression and low perceived recovery quality.
